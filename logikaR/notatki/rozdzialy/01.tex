\chapter{Formalizacja matematyki}

\section{02.10.2025}{Uproszczony model rzeczywistości matematycznej: struktura I rzędu}

\subsection{Model języka i język struktury modelu}

\begin{definition}{model}{}
  \buff{Model} to struktura matematyczna składająca się z
  \begin{itemize}
    \item niepustego zbioru będącego \acc{uniwersum} $A\neq \emptyset$,
    \item \acc{funkcji} $f_1,..., f_k$ o arności $n_i$ (tzn. $f_i:A^{n_i}\to A$),
    \item \acc{relacji} (predykatów) w $A$, $P_1,..., P_n$, gdzie $P_i\subseteq A^{n_i}$,
    \item \acc{stałych} z $A$ $c_1,..., c_l\in A$.
  \end{itemize}
\end{definition}

Zapisujemy
$$\Mm=(A; f_1,...,f_k; P_1,...,P_n; c_1,...,c_l)$$
gdzie $k,n,l$ to liczby kardynalne, zazwyczaj skończone (tzn. $k,n,l\in\N$).

\begin{example}[m]
\item Jeśli $n=0$, czyli nie mamy relacji, to \Mm{} jest strukturą algebraiczną. Weźmy na przykład grupę. Grupa jest zdefiniowana jako zbiór $G$ z wyróżnionym elementem neutralnym $e$, operacją mnożenia $\cdot$ oraz brania elementu odwrotnego $^{-1}$. Operacje to funkcje, a element neutralny to stała. Sam zbiór $G$ to z kolei uniwersum, czyli mamy model:
  $$(G; \cdot, ^{-1};\;;e)$$
\item Rodzina zbiorów $V$ z relacją należenia $\in$ jest modelem z jedną relacją, ale bez funkcji i bez stałych:
  $$(V;\;;\in;\;)$$
\end{example}

Symbole oznaczające funkcje, relacje, stałe będziemy od ich znaczenia odróżniać przez podkreślenie:
\begin{itemize}
  \item $\underline{f_i}$, $\underline{P_j}$, $\underline{c_t}$ to symbole, 
  \item natomiast $f_i$, $P_J$, $c_t$ to funkcja, relacja, stała.
\end{itemize}

\begin{definition}{język}{}
  Język
  $$L=\{\underline{f_1},...,\underline{f_k};\underline{P_1},..., \underline{P_n}; \underline{c_1},...,\uu{c_l}\}$$
  składa się z symboli: funkcyjnych, relacyjnych, stałych wraz z przypisanymi tym symbolom arnościami, tzn. $f_j$ to symbol funkcjsi $n_i$-argumentowej etc.
\end{definition}

Język jak wyżej jest nazywany językiem struktury \Mm{}, typem podobieństwa \Mm{}, \buff{sygnaturą \Mm{}}. Z kolei \Mm{} jest modelem dla $L$.

Szerzej będziemy dla \Mm{} - modelu dla $L$ - pisać
$$(\Mm; \uu{f_1}^\Mm, ..., \uu{f_k}^\Mm; \uu{P_1}^\Mm,...,\uu{P_n}^\Mm; \uu{c_1}^\Mm,...,\uu{c_l}^\Mm)$$
gdzie $\uu{f_i}^\Mm$ oznacza interpretację symbolu $\uu{f_1}^\Mm$ w kontekście modelu \Mm.

\begin{remark}{}{}
  Dla dowolnego języka $L$ istnieje wiele struktur \Mm.
\end{remark}

Mając dany język $L$ mówimy/piszemy w nim przy pomocy
\begin{itemize}
  \item symbolów języka,
  \item symboli logicznych $\land$, $\lor$, $\neg$, $\to$, $\lr$ (!!! $\implies$ oraz $\iff$ będą dla nas elementami metajęzyka !!!), $\forall$, $\exists$, $=$,
  \item zmiennych, np. $x_i$ dla $i\in \N$, $y$, $z$,
  \item oraz symboli pomocniczych takich jak nawiasy, przecinki etc.
\end{itemize}

\begin{remark}{}{}
  Spójniki można ograniczyć do $\land$, $\neg$ i kwantyfikatora $\exists$. Całą resztę spójników można zdefiniować jako macra przy pomocy tych trzech, np.
  $$p\lor q\quad :=\quad \neg(\neg p\land \neq q)$$
\end{remark}

\textbf{Wyrażenia języka $L$:}
\begin{enumerate}[label=\alph*)]
  \item \buff{wyrażenia nazwowe} (termy) należą do $\mathcal{T}_L$ i są definiowane rekurencyjnie:
    \begin{itemize}
      \item zmienna, symbol stałej należą do $\mathcal{T}_L$ i nazywają się \acc{termami atomowymi}
      \item jeśli $\tau_1$, ..., $\tau_n\in \mathcal{T}_L$, a $\uu{f}$ jest symbolem $n$-argumentowej funkcji z $L$, to $\uu{f}(\tau_1,...,\tau_n)\in \mathcal{T}_L$ i nazywa się \acc{termem złożonym}. 
    \end{itemize}
  \item \buff{formuły} oznaczamy $\mathcal{F}_L$ i definiujemy rekurencyjnie w następujący sposób
    \begin{itemize}
      \item dla wszystkich termów $\tau_1,...,\tau_n$ zachodzi $(\tau_1=\tau_2)\in \mathcal{F}_L$ oraz dla $n$-argumentowego symbolu relacji $\uu{P}_j$: $\uu{P}_j(\tau_1,...,\tau_n)\in  \mathcal{F}_L$ - to są \acc{formuły atomowe},
      \item $\phi,\psi\in\mathcal{F}_L\implies (\neg \phi), (\phi\land\psi)\in \mathcal{F}_L$

        $\phi\in \mathcal{F}_L$ $\implies (\exists\;r\;\phi),(\forall\;r\phi)\in\mathcal{F}_L$ ($r$ występujące w wyrażeniach nazywamy zmiennymi)
    \end{itemize}
\end{enumerate}

\subsection{Zdania w języku}

Niech $\phi\in \mathcal{F}_L$ będzie formułą w której występuje, co najmniej raz, zmienna $v$. Jeśli pewne wystąpienie $v$ w $\phi$ jest w zasięgu pewnego kwantyfikatora $Q_v\in\{\forall,\;\exists\}$, to spośród wszystkich wystąpień $Q_v$ w $\phi$ w których zasięgu jest $v$ wybieramy to najbardziej na prawo i mówimy, że to \buff{$Q_v$ wiąże dane wystąpienie $v$} w $\phi$. Na przykład
\begin{center}
  \begin{tikzpicture}
    \node at (0,0) {$\forall\;x\;\exists\;y\;(x\in y\;\land\;\forall\;x\;(x\in y \to x=y))$};
    \draw[->] (-2.95, -.1) -- (-2.95, -.5) -- (2.4, -.5) -- (2.3, -.1);
    \draw[->, green] (.2, -.1) -- (.2, -.7) -- node[midway, below] {wiąże} (2.1, -.7) -- (2.2, -.1);
  \end{tikzpicture}
\end{center}
Jeśli nie ma kwantyfikatora $Q_v$ jak wyżej, to wystąpienie $v$ w $\phi$ jest \buff{wolne}.

Zapis $\phi(v_1,...,v_n)$ oznacza, że wszystkie wolne zmienne w $\phi$ są wśród $v_1$, ..., $v_n$.

\begin{definition}{zdanie}{}
  Formalne zdanie w języku $L$ to formuła niezawierająca zmiennych wolnych.
\end{definition}

\buff{Powstaje pytanie co to znaczy, że formuła z $L$ jest prawdziwa w strukturze \Mm{} dla $L$?}

Niech \Mm{} będzie modelem dla $L=\{\uu{f}_i,...,\uu{P}_j,...,\uu{c}_t,...\}$, $\{\uu{a}\;:\;a\in \Mm\}$ będzie zbiorem nowych symboli stałych. Rozważmy nowy język $L(\Mm)=L\cup \{\uu{a}\;:\;a\in \Mm\}$.
Termy stałe $\tau$ z $L(\Mm)$ interpretujemy w \Mm{} w następujący sposób:
\begin{itemize}
  \item jeśli $\tau$ jest symbolem stałym w $L$, to $\tau^\Mm\in\Mm$ jest interpretacją $\uu{c}_i^\Mm$
  \item jeśli $\tau$ {\color{red}CO TU SIE WYTENTEGOWUYJE}
\end{itemize}

\begin{definition}{}{}
  $\Mm\models\phi$ oznacza, że $\phi$ jest prawdziwe/spełniane w \Mm.
\end{definition}

\begin{enumerate}[label=\alph*)]
  \item zdania atomowe:
    \begin{itemize}
      \item $\Mm\models \tau_1=\tau_2\iff\uu{\tau}_1^\Mm=\uu{\tau}_2^\Mm$
      \item $\Mm\models \uu{P}_j(\tau_1,..., \tau_n)\iff (\uu{\tau}_1^\Mm,...,\uu{\tau}_n^\Mm)\in\uu{P}_j^\Mm$
    \end{itemize}
  \item zdania złożone:
    \begin{itemize}
      \item $\Mm\models\phi\;\land\;\phi\iff\Mm\models\phi\text{ oraz }\Mm\models\psi$
    \end{itemize}
\end{enumerate}




