\chapter{Formalizacja matematyki}

\section{02.10.2025}{Uproszczony model rzeczywistości matematycznej: struktura I rzędu}

\subsection{Model języka i język struktury modelu}

\begin{definition}{model}{}
  \buff{Model} to struktura matematyczna składająca się z
  \begin{itemize}
    \item niepustego zbioru będącego \acc{uniwersum} (dziedzina) $A\neq\emptyset$,
    \item \acc{funkcji} $f_1,..., f_k$ o arności $n_i$ (tzn. $f_i:A^{n_i}\to A$),
    \item \acc{relacji} (orzeczników, predykatów) w $A$, $P_1,..., P_n$, gdzie $P_i\subseteq A^{n_i}$,
    \item \acc{stałych} z $A$ $c_1,..., c_l\in A$.
  \end{itemize}
\end{definition}

Zapisujemy
$$\Mm=(A; f_1,...,f_k; P_1,...,P_n; c_1,...,c_l)$$
gdzie $k,n,l$ to liczby kardynalne, zazwyczaj skończone (tzn. $k,n,l\in\N$).

\begin{konwencja}{}{}
  $$(\Mm:=)|\Mm|:=A$$
\end{konwencja}

\begin{example}[m]
\item Jeśli $n=0$, czyli nie mamy relacji, to \Mm{} jest \acc{strukturą algebraiczną} (algebrą ogólną). Weźmy na przykład grupę. Grupa jest zdefiniowana jako zbiór $G$ z wyróżnionym elementem neutralnym $e$, operacją mnożenia $\cdot$ oraz brania elementu odwrotnego $^{-1}$. Operacje to funkcje, a element neutralny to stała. Sam zbiór $G$ to z kolei uniwersum, czyli mamy model:
  $$(G; \cdot, ^{-1};\;;e)$$
\item Rodzina zbiorów $V$ z relacją należenia $\in$ jest modelem z jedną relacją, ale bez funkcji i bez stałych:
  $$(V;\;;\in;\;)$$
  Taką strukturę nazywamy \acc{strukturą relacyjną}.
\end{example}

Symbole oznaczające funkcje, relacje, stałe będziemy od ich znaczenia odróżniać przez podkreślenie:
\begin{itemize}
  \item $\underline{f}_i$, $\underline{P}_j$, $\underline{c}_t$ to symbole, 
  \item natomiast $f_i$, $P_J$, $c_t$ to funkcja, relacja, stała.
\end{itemize}

\begin{definition}{język}{}
  Język
  $$L=\{\underline{f}_1,...,\underline{f}_k;\underline{P}_1,..., \underline{P}_n; \underline{c}_1,...,\uu{c}_l\}$$
  składa się z symboli: funkcyjnych, relacyjnych, stałych wraz z przypisanymi tym symbolom arnościami, tzn. $f_j$ to symbol funkcji $n_i$-argumentowej etc.
\end{definition}

Język jak wyżej jest nazywany językiem struktury \Mm{}, typem podobieństwa \Mm{}, \buff{sygnaturą \Mm{}}. Z kolei \Mm{} jest modelem dla $L$.

\begin{konwencja}{}{}
Szerzej będziemy dla \Mm{} - modelu dla $L$ - pisać
$$(\Mm; \uu{f_1}^\Mm, ..., \uu{f_k}^\Mm; \uu{P_1}^\Mm,...,\uu{P_n}^\Mm; \uu{c_1}^\Mm,...,\uu{c_l}^\Mm)$$
gdzie $\uu{f_i}^\Mm$ oznacza interpretację symbolu $\uu{f_1}^\Mm$ w kontekście modelu \Mm.
\end{konwencja}

\begin{remark}{}{}
  Dla dowolnego języka $L$ istnieje wiele struktur \Mm.
\end{remark}

\begin{example}
  Język grup
\end{example}

Mając dany język $L$ mówimy/piszemy w nim przy pomocy
\begin{itemize}
  \item symbolów języka,
  \item symboli logicznych $\land$, $\lor$, $\neg$, $\to$, $\lr$ (!!! $\implies$ oraz $\iff$ będą dla nas elementami metajęzyka !!!), $\forall$, $\exists$, $=$,
  \item zmiennych, np. $x_i$ dla $i\in \N$, $y$, $z$,
  \item oraz symboli pomocniczych takich jak nawiasy, przecinki etc.
\end{itemize}

\begin{remark}{}{}
  Spójniki można ograniczyć do $\land$, $\neg$ i kwantyfikatora $\exists$. Całą resztę spójników można zdefiniować jako macra przy pomocy tych trzech, np.
  $$p\lor q\quad :=\quad \neg(\neg p\land \neq q)$$
\end{remark}

\textbf{Wyrażenia języka $L$:}
\begin{enumerate}[label=\alph*)]
  \item \buff{wyrażenia nazwowe} (termy) należą do $\mathcal{T}_L$ ($\text{Term}_L$) i są definiowane rekurencyjnie:
    \begin{itemize}
      \item zmienna, symbol stałej należą do $\mathcal{T}_L$ i nazywają się \acc{termami atomowymi}
      \item jeśli $\tau_1$, ..., $\tau_n\in \mathcal{T}_L$, a $\uu{f}$ jest symbolem $n$-argumentowej funkcji z $L$, to $\uu{f}(\tau_1,...,\tau_n)\in \mathcal{T}_L$ i nazywa się \acc{termem złożonym}. 
    \end{itemize}
  \item \buff{formuły} oznaczamy $\mathcal{F}_L$ ($\text{Form}_L$) i definiujemy rekurencyjnie w następujący sposób
    \begin{itemize}
      \item dla wszystkich termów $\tau_1,...,\tau_n$ zachodzi $(\tau_1=\tau_2)\in \mathcal{F}_L$ oraz dla $n$-argumentowego symbolu relacji $\uu{P}_j$: $\uu{P}_j(\tau_1,...,\tau_n)\in  \mathcal{F}_L$ - to są \acc{formuły atomowe},
      \item $\phi,\psi\in\mathcal{F}_L\implies (\neg \phi), (\phi\land\psi)\in \mathcal{F}_L$

        $\phi\in \mathcal{F}_L$ $\implies (\exists\;r\;\phi),(\forall\;r\phi)\in\mathcal{F}_L$ ($r$ występujące w wyrażeniach nazywamy zmiennymi) 
        
        Formuły z tego punktu nazywamy \acc{formuły złożone}.
    \end{itemize}
\end{enumerate}

\buff{Hierarchia symboli} logicznych umożliwia pomijanie nawiasów:
\begin{enumerate}
  \item symbole matematyczne
  \item kwantyfikatory
  \item negacja $\neg$
  \item $\land$, $\lor$
  \item $\to$, $\lr$
\end{enumerate}

\subsection{Zdania w języku}

Niech $\phi\in \mathcal{F}_L$ będzie formułą w której występuje, co najmniej raz, zmienna $v$. Jeśli pewne wystąpienie $v$ w $\phi$ jest w zasięgu pewnego kwantyfikatora $Q_v\in\{\forall,\;\exists\}$, to spośród wszystkich wystąpień $Q_v$ w $\phi$ w których zasięgu jest $v$ wybieramy to najbardziej na prawo i mówimy, że to \buff{$Q_v$ wiąże dane wystąpienie $v$} w $\phi$. Na przykład
\begin{center}
  \begin{tikzpicture}
    \node at (0,0) {$\forall\;x\;\exists\;y\;(x\in y\;\land\;\forall\;x\;(x\in y \to x=y))$};
    \draw[->] (-2.95, -.1) -- (-2.95, -.5) -- (2.4, -.5) -- (2.3, -.1);
    \draw[->, green] (.2, -.1) -- (.2, -.7) -- node[midway, below] {wiąże} (2.1, -.7) -- (2.2, -.1);
  \end{tikzpicture}
\end{center}

Jeśli nie ma kwantyfikatora $Q_v$ jak wyżej, to wystąpienie $v$ w $\phi$ jest \buff{wolne}. Popatrzmy na przykład na formułę
$$\exists {\color{red}y}\;x\in {\color{red}y}\;\land\;\forall\;x\;(x\in {\color{blue}y}\to x={\color{blue}y}).$$
Kwantyfikator z czerwonym $y$ wiąże czerwony $y$, a niebieskie $y$ pozostają wolne. Dodając nawias pierwszy kwantyfikator wiąże już wszystkie występienia $y$:
$$\exists {\color{red}y}\;(x\in {\color{red}y}\;\land\;\forall\;x\;(x\in {\color{red}y}\to x={\color{red}y})).$$

\begin{konwencja}{}{}
  Zapis $\phi(v_1,...,v_n)$ oznacza, że wszystkie wolne zmienne w $\phi$ są wśród $v_1$, ..., $v_n$ (ale niekoniecznie wszystkie $v_i$ są zmiennymi wolnymi).
\end{konwencja}

\begin{definition}{zdanie}{}
  Formalne zdanie w języku $L$ to formuła niezawierająca zmiennych wolnych.
\end{definition}

Tarski podał na początku XX wieku definicję prawdy. Nieco później udowodnił twierdzenie, że nie da się prawdy zdefiniować.

\buff{Powstaje pytanie co to znaczy, że formuła z $L$ jest prawdziwa w strukturze \Mm{} dla $L$?}

Niech \Mm{} będzie modelem dla $L=\{\uu{f}_i,...,\uu{P}_j,...,\uu{c}_t,...\}$ oraz $\{\uu{a}\;:\;a\in \Mm\}$ niech będzie zbiorem nowych symboli stałych. Rozważmy nowy, większy język 
$$L(\Mm)=L\cup \{\uu{a}\;:\;a\in |\Mm|\},$$
którego modelem nadal jest \Mm{} ($\uu{a}^\Mm:=a$).

Termy stałe $\sigma^\Mm\in |\Mm|$ z $L(\Mm)$ interpretujemy w \Mm{} w następujący sposób:
\begin{itemize}
  \item jeśli $\uu{\tau}$ jest symbolem stałym $\uu{c}_i$ w $L$, to $\tau^\Mm\in\Mm$ jest interpretacją $\uu{c}_i^\Mm$
  \item jeśli $\tau=\uu{f}_i(\uu{\tau}_1,...,\uu{\tau}_n)$ jest termem złożonym, to $\sigma^\Mm=\uu{f}_i^\Mm(\uu{\tau}_1^\Mm,...,\uu\tau_n^\Mm)$. 
\end{itemize}

\begin{konwencja}{}{}
  $\Mm\models\phi$ oznacza, że $\phi$ jest prawdziwe/spełniane w \Mm.
\end{konwencja}

\begin{enumerate}[label=\alph*)]
  \item zdania atomowe:
    \begin{itemize}
      \item $\Mm\models \tau_1=\tau_2\iff\tau_1^\Mm=\tau_2^\Mm$
      \item $\Mm\models \uu{P}_j(\tau_1,..., \tau_n)\iff (\tau_1^\Mm,...,\tau_n^\Mm)\in\uu{P}_j^\Mm$
    \end{itemize}
  \item zdania złożone:
    \begin{itemize}
      \item $\Mm\models\phi\;\land\;\phi\iff\Mm\models\phi\text{ oraz }\Mm\models\psi$
      \item $\Mm\models\neg\phi\iff $ nieprawda, że $\Mm\models \phi$ (oznaczamy $\Mm\not\models\phi$)
      \item $\Mm\models\exists\;v\;\phi$ (jest tym samym co $\exists\;v\;\phi(v)$, bo zakładamy, że nie ma innych zmiennych wolnych w $\phi$, bo to jest zdanie) $\iff \Mm\models\underset{=\phi(\uu{a})}{\phi(v/\uu{a})}$ dla pewnego $a\in|\Mm|$, gdzie $\phi(\uu{a})$ jest formułą powstającą z $\phi$ przez zastąpnienie każdego wolnego wystąpienia $v$ w $\phi$ przez $\uu{a}$.
    \end{itemize}
\end{enumerate}

To daje nam, że $\Mm\models\phi$ dla każdego $\phi\in\mathcal{F}_L$.

Kwantyfikatujemy po zmiennych, nie po podzbiorach uniwersum, bo to jest logika I rzędu. Zbiory tworzymy w umyśle, postrzegamy indywidua. Teorie wyższych rzędów nie są absolutne.

% \buff{Atomizm logiczny} - każde zdanie można zredukować do zdań atomowych, których już dalej się rozbić nie da. 

% Można tutaj od razu udowodnić, że zachodzi prawo wyłączonego środka.

% Spełnianie dla formuł $\phi(v_1,...,v_n)$ języka $L$:
% \begin{itemize} $(a_1,...,a_n)\subseteq |\Mm| $ spełnia $\phi$ jak wyżej w $\Mm$, jeśli $\Mm\models\phi(\uu{a}_1,...,\uu{a}_n)$
%   \item uniwersalne domknięcie $\phi(v_1,...,v_n)$ to formuła
%     $$\overline{\phi}=\forall\;v_1,...,\forall\;v_n\;\phi(v_1,...,v_n)$$
%     z dokładnością do równoważności formuł jest jednoznacznie zdefiniowane $\Mm\models\phi\iff\Mm\models\overline\phi$
% \end{itemize}

\begin{definition}{spełnianie formuły}{}
  Krotka $\langle a_1,...,a_n\rangle$ elementów z uniwersum $\Mm$ \buff{spełnia formułę} $\phi$ języka, gdy $\Mm\models\phi(\uu{a}_1,...,\uu{a}_n)$.
\end{definition}

\begin{definition}{uniwersalne domknięcie}{}
  Dla formuły $\phi$ języka $L$ jej \buff{uniwersalne domknięcie} $\phi$ to formuła
  $$\overline{\phi}:=\forall\;x_1\;\forall\;x_2\;...\forall\;x_n\;\phi$$
\end{definition}
W następnej części wykładu pokażemy, że 
$$\Mm\models\phi\iff\Mm\models\overline{\phi}$$

Powyższe zagadnienia mówią, że domyślny kwantyfikator to kwantyfikator $\forall$. To ma spełnienie w naturalnej matematyce, np. przemienność dodawania $(\R, +)\models\;x+y=y+x$ gdzie pomijamy $\forall\;x\;\forall\;y$.

\subsection{Tautologie}

\begin{definition}{tautologia KRL}{}
  Niech $\phi\in\mathcal{F}_L$. Wtedy $\phi$ jest \buff{tautologią} klasycznego rachunku logicznego [KRL], gdy jest zawsze prawdziwe. Zapisujemy to $\models\phi$, co oznacza $\forall\;\Mm$ modelu $L$ $\Mm\models\phi$.
\end{definition}

\buff{Jak rozpoznać, czy $\models\phi$?} Ogólnie jest to pytanie nierozstrzygalne (twierdzenie G\"odla).

Niektóre tautologie łatwo rozpoznać, jak przykłady niżej.
\begin{example}[m]
  \item $x=x$
  \item $=$ jest relacją równoważności
\end{example}

Przygodę z tautologiami rozpoczynamy od \buff{tautologicznych formuł zdaniowych} (czyli schematów).

Niech $Z=\{p_0, p_1,...,p_n, r, q,...\}$ będzie zbiorem zmiennych zdaniowych. Zbiór formuł zdaniowych $S=S_Z$ nad $Z$ definiujemy w następujący sposób:
\begin{itemize}
  \item $v\in Z\implies v\in S$
  \item $\alpha, \beta\in S\implies \neg\alpha, \alpha\;\land\;\beta\in S$
  \item skrótowo: $\lor, \to, \lr\in S$
\end{itemize}

\begin{definition}{wartościowanie}{}
  \buff{Wartościowanie logiczne formuł zdaniowych} to dowolna funkcja $v:S\to \{0,1\}$ taka, że dla każdych $\alpha,\beta\in S$ 
  \begin{itemize}
    \item $v(\neg\alpha)=1-v(\alpha)$
    \item $v(\alpha\;\land\;\beta)=\min\{v(\alpha), v(\beta)\}$
  \end{itemize}
\end{definition}

\begin{definition}{tautologia KRZ}{}
  Powiemy, że $\alpha\in S$ jest \buff{tautologią} ($\models\alpha$) gdy dla każdego wartościowania $v$ zachodzi $v(\alpha)=1$.
\end{definition}

Na przykład: $\models\neg(\alpha\;\land\;\neg\alpha)$ dla każdego $\alpha\in S$

Istnieje algorytm rozstrzygający, czy $\models \alpha$ dla $\alpha\in S$ (metoda 0-1): wartość $v(\alpha)$ zależy tylko od $v(x)$ dla zmiennych zdaniowych $x$ w $\alpha$.

\begin{definition}{przykład formuły}{}
  Załóżmy, że $\alpha\in S$ jest zbudowany ze zmiennych $p_1,..., p_n\in Z$ oraz $\phi_1,..., \phi_n\in\mathcal{F}_L$ i $\phi\in\mathcal{F}_L$ powstaje z $\alpha$ przez zastąpienie $p_i$ przez $\phi_i$ wszędzie w $\alpha$ (zapisujemy $\phi=\alpha(p_1/\phi_1,...,p_n/\phi_n)$. Mówimy, że $\phi$ jest \buff{jest przykładem formuły $\alpha$}. 
\end{definition}

Weźmy na przykład formułą $\alpha=p_1\;\land\;p_2$, wtedy $\phi=\phi_1\;\land\;\phi_2$ jest przykładem formuły $\alpha$.

\begin{theorem}{}{}
  Jeśli $\models\alpha$ i $\phi$ jest przykładem $\alpha$ to $\models\phi$.
\end{theorem}
\begin{proof}
  Ćwiczenie
\end{proof}

{\color{red}\large TUTAJ KONCZE REVISIONS}

\begin{definition}{}{}
  Reguła wnioskowania składa się z:
  \begin{itemize}
    \item przesłanek $\phi_1,....,\phi_n\in\mathcal{F}_L$
    \item tezy $\phi$.
  \end{itemize}
  Zapisujemy 
  $$\frac{\phi_1,...,\phi_n}{\phi}$$

  Reguła wnioskowania jest poprawna, kiedy nie prowadzi nas na manowce. 
  \begin{enumerate}
    \item $\frac{\phi_1,...,\phi_n}{\phi}$ jest \buff{poprawna}, jeśli dla każdego modelu $\Mm$ dla $L$ $\Mm\models\phi_1\;\land...\land\;\phi_n\implies \Mm\models\phi$
    \item $\frac{\alpha_1,...,\alpha_n}{\alpha}$ jest poprawna, jeśli dla każdego wartościowania $v$ 
  \end{enumerate}
\end{definition}

\begin{example}[m]
  \item Modus Ponens (reguła odrywania, cut rule) 
    $$\frac{\alpha,\alpha\to \beta}{\beta}$$
  \item reguła generalizacji ($\forall\;$-rule)
    $$\frac{\phi}{\forall\;v\;\phi(v)}$$
\end{example}

KRL - $\phi$ etc, KRZ - $\alpha$, $\beta$ etc

Aksjomatyczne ujęcie KRL:
\begin{itemize}
  \item[(A0)] przykład dowolnej tautologii KRZ
  \item[(A1)] $\forall\;v\;(\phi\to\psi)\to(\phi\to\forall\;v\;\psi)$, gdy $v$ nie jest wolna w $\phi$
  \item[(A2)] $\forall\;v\;\phi\to\phi(v/t)$ (to jest poprawny zapis, ale dla pewności: $(\forall\;v\;\phi)\to\phi(v/t)$)
    $t$ to jest term podstawiany za każde wolne wystąpienie $v$ w $\phi$ pod warunkiem, że żadne z takich wystąpień nie jest w zasięgu kwantyfikatora wiążącego zmienną występującą w termie $t$.

    Zastrzeżenie jest istotne: $\phi:\exists\;y\;x\neq y$, $t=y$, wtedy $\phi(x/t)=\exists\;y\;y\neq y$ i mamy
    $$\not\models\forall\;x\phi\to\phi(x/t)=\forall\;\exists\;y\;x\neq y\to \exists\;y\;y\neq y$$
\end{itemize}

Aksjomaty równości ($v_i$ - wolne zmienne):
\begin{itemize}
  \item $v_1=v_1$
  \item $v_1=v_2\to t(...v_1...)=t(...v_1/v_2...)$ 
  \item $v_1=v_2\to (\phi(...v_1...)\to \phi(...v_2...))$ gdzie $v_1$ to wolne wystąpienie w $\phi$ niebędące w zasięgu kwantyfikatora wiążącego $v_2$.
\end{itemize}

\begin{definition}{}{}
  Dowodliwość w KRL: dla $X\subseteq \mathcal{F}_L$ oraz $\phi\in\mathcal{F}_L$

  $X\vdash \phi$ $\iff$ istnieje ciąg formuł (dowód formalny) $\alpha_1$, .., $\alpha_n=\phi$ takie, że dla każdego $i\leq n$
  \begin{enumerate}
    \item $\alpha_1\in X\;\lor\;\alpha_i$ jest aksjomatem KRL
    \item $\alpha_i$ wynika z $\{\alpha_1,...,\alpha_{i-1}\}$ na mocy Morus Ponens lub $\forall$-rule, tzn. $\exists\;j,t<i\;\alpha_t=\alpha_j\to \alpha_i$
      $$\frac{\alpha_j,\alpha_j\to\alpha_i}{\alpha_i}$$
      lub $\exists\;j<i\;\alpha_i=\forall\;v\;\alpha_j$
      $$\frac{\alpha_j}{\forall\;v\alpha_j}$$
  \end{enumerate}
\end{definition}

\begin{konwencja}{}{}
  $\vdash \phi$ gdy $\emptyset\vdash\phi$

  Takie $\phi$ nazywamy \buff{tezą} KRL
\end{konwencja}

\begin{remark}{}{}
  $X\vdash \phi$ $\iff$ $\exists\;X_0\subseteq X$ skończony $X_0\vdash \phi$
\end{remark}

\begin{example}
  Pokażemy, że $\vdash \forall\;x\;\phi\to\exists\;x\;\phi$, to znaczy
  $$\vdash \forall\;x\;\phi\to \neg\;\forall\;x\;\neg\phi$$
  
  \begin{enumerate}
    \item $\alpha_1:\forall\;x\;\phi\to \phi(x/y)$ (A2)
    \item $\alpha_2:\forall\;x\;\neg\phi\to\neg\phi(x/y)$ (A2)
    \item $\alpha_3:\alpha_2\to (\phi(x/y)\to \neg\forall\;x\;\neg\phi)$ (A0 $(p\to\neg q)\to(q\to\neg p)$)
    \item $\alpha_4:\phi(x/y)\to\neg\forall\;x\;\neg\phi$ ($\alpha_2,\alpha_3$ oraz Modus Ponens)
    \item $\alpha_5:\alpha_1\to(\alpha_4\to \beta)$, gdzie $\beta=\forall\;x\;\phi\to\neg\forall\;x\;\neg\phi$ (A0, przechodność)
    \item $\alpha_6:\beta$ (Modus Ponens)
  \end{enumerate}
\end{example}

\begin{example}
  $\phi:x=x$
  $$\vdash\forall\;x\;x=x\to\exists\;x\;x=x$$
  jest fałszywe w dziedzinie pustej, czyli modelu
  $$(\emptyset; f_1,...,f_n; P_1,...,P_j;)$$
  gdzie nie ma stałych. Czyli coś z aksjomatów musi zawodzić w dziedzinie pustej - pytanie co?
\end{example}

\begin{center}
  \buff{Od tej pory modele są niepuste}
\end{center}

\begin{remark}{}{}
  Poprawność (soundness) KRL
  $$\vdash\phi\implies \models\phi$$
\end{remark}
\begin{proof}
  Ćwiczenie
\end{proof}

\begin{theorem}{G\"odela o pełności KRL}{}
  $$\models\phi\implies\vdash\phi$$
\end{theorem}

\begin{definition}{}{}
  Dla $X\subseteq\mathcal{F}_L$ zbiór konsekwencji $X$ to $\Cn(X)=\{\phi\in\mathcal{F}_L\;:\;X\vdash \phi\}$
  \begin{itemize}
    \item $X$ jest teorią, gdy $X=\Cn(X)$
    \item $A$ jest zbiorem aksjomatów teorii $X$, gdy $X=\Cn(A)$
    \item zbiory $X$, $Y$ są równoważne, gdy $\Cn(X)=\Cn(Y)$ co równoważnie można powiedzieć: $X\vdash Y$ i $Y\vdash X$
  \end{itemize}
\end{definition}

\begin{fact}{}{}
  $\phi$ oraz $\overline{\phi}$ są równoważne
\end{fact}
\begin{proof}
  $\phi\vdash\overline{\phi}$ to $\forall$-reguła zastosowana tyle razy ile użyliśmy kwantyfikatorów w $\overline{\phi}$

  $\overline{\phi}\vdash\phi$ to z twierdzenia o dedukcji (zadanie) $\overline{\phi}\vdash\phi\iff\vdash\overline{\phi}\to\phi$ (drugie to wynik z A2) 
\end{proof}

\begin{conclusion}{}{}
  $X\subseteq\mathcal{F}_L$

  $X$ oraz $\{\overline{\phi}\;:\;\phi\in X\}$ są równoważne
\end{conclusion}

\begin{definition}{}{}
  $X$ jest sprzeczny, jeżeli $X\vdash\phi$ oraz $X\vdash\neg\phi$ dla pewnego zdania $\phi\in\mathcal{F}_L$
\end{definition}

W przeciwnym razie $X$ jest niesprzeczny.

\begin{definition}{}{}
  \begin{enumerate}
    \item $\Mm\models X\iff \forall\;\phi\in X\;\Mm\models\phi$ (\Mm jest modelem $X$)
    \item $X$ jest zupełny $\iff$ $\forall$ zdania $\phi$ ($X\vdash \phi$ lub $x\vdash\neg\phi$)
    \item $X$ jest rozstrzygalny $\iff$ istnieje algorytm rozstrzygający, czy $X\vdash\phi$
  \end{enumerate}
\end{definition}

\begin{example}[m]
\item teoria $\Mm$ $\Th(\Mm):=\{\phi\;:\;M\models\phi\}$ jest niesprzeczna i zupełna
\item $\emptyset$ jest niesprzeczny
\end{example}

\begin{theorem}{G\"odela o istnieniu modelu}{}
  Jeśli $S$ jest niesprzecznym zbiorem zdań, to $S$ ma model.
\end{theorem}

\begin{proof}
  (by Leon Henkin)

  Dla przypadków, gdy $L$ i $S$ są przeliczalne, ogólny dowód jest bez większej idei, ale używa "ble ble kombinatoryczne z liczb kardynalnych".

  $L'=L\cup\{c_n\;:\;n\in\N\}$, gdzie $c_n$ to nowe w stosunku do $L$ symbole stałe

  $\{\phi_n\;:\;n\in\N\}$ to numeracja formuł $\mathcal{F}_L(x)$ z co najwyżej jedną wolną zmienną $x$

  Niech $f:\omega\to\omega$ (czyli $\N\to\N$) rosnąca taka, że $c_{f(n)}$ nie występuje w formułach $\phi_0(x),...,\phi_n(x)$

  $S_n=S\cup\{\exists\;x\;\phi_i(x)\to\phi_i(x/c_{f(i)}\;:\;i<n\}$ (aksjomat postaci jaka w $\{\}$ nazywa się aksjomatem Henkina

  $S=S_0\subseteq S_1\subseteq...$ to ciąg rosnący
  
  $S_\omega=\bigcup S_n$

  FAKT: zbiór $S_\omega$ jest niesprzeczny

  Dowód nie wprost: gdyby był sprzeczny, to sprzeczność wynikałaby ze skończonego podzbioru, czyli pewne $S_n$ jest sprzeczne, czyli możemy wybrać $n$ najmniejsze takie, że $S_{n+1}=S_n\cup\{\exists\;x\;\phi_n(x)\to\phi_n(x/c_{f(n)})\}$ jest sprzeczne

  ponieważ $n$ jest najmniejsze, to $S_n$ nie jest sprzeczne

  $H$ niech będzie zdaniem w $\{\}$

  $$S_n\cup\{H\}\vdash \alpha\;\land\;\neg\alpha$$
  dla pewnego $\alpha$
  z tego wiemy, że
  $$S_n\vdash(\exists\;x\;\phi_n(x)\to\phi_n(c_{f(n)})\to\alpha\;\land\;\neg\alpha$$
  A0 daje nam, że 
  $$S_n\vdash \neg(\exists\;x\;\phi_n(x)\to\phi_n(c_{f(n)}))$$
  znowu korzystając z A0 dostajemy
  $$S_n\vdash\exists\;x\;\phi_n(x)\;\land\;\neg\phi_n(c_{f(n)})$$
  $$S_n\vdash\exists\;x\;\phi_n(x)\text{ i } \S_n\vdash\neg\phi_n(c_{f(n)})$$
  w dowodzie po prawej od i $c_{f(n)}$ zastępujemy nową zmienną $y$

  po tej zamianie powstaje dowód $S_n\vdash \neg\phi_n(y)$, korzystamy z $\forall$ żeby dostać $S_n\vdash\forall\;y\neq\phi_n(y)$ z A2 dla $t=x$ dostajemy $S_n\vdash\neg\phi_n(x)$, z $\forall$-reguły mamy $S_n\vdash\forall\;x\;\neg\phi_n(x)$, ale nadal prawdą jest, że $\S_n\vdash\exists\;x\;\phi_n(x)$

  Mamy skońcozny dowód, że $S_\omega$ jest niesprzeczny

  Twierdzenie Lidenbauma(???) jeśli mamy niesprzeczny zbiór formuł/zdań to możemy go rozszerzyć do maksymalnego niesprzecznego zbioru formuł/zdań, który jest zupełny i niesprzeczny zbiór formuł/zdań
  
  Korzystamy, żeby dostać $S'\supseteq S_\omega$, który jest zupełny i niesprzecznym zbiorem zdań w $L'$

  $S'$ opisuje strukturę modelu na $\{c_n\;:\;n<\omega\}$

  mamy relację równoważności na zbiorze $C=\{c_n\;:\;n<\omega\}$: $c_n\sim c_m\iff S'\vdash c_n=c_m$
  \begin{itemize}
    \item symetria: aksjomat równości
  \end{itemize}

  Niech $\Mm=\{c_n/\sim\;:\;n\in\omega\}=C/\sim$
  $L'$ - struktura na $M$
  \begin{enumerate}
    \item $P_i^\Mm(c_{i_1}/\sim,...,c_{i_n}/\sim)\iff S'\vdash P_i(c_{i_1},...,c_{i_n})$
    \item $f^\Mm_i(c_{i_1}/\sim,...,c_{i_k}/\sim)=c_{i_n}/\sim\iff S'\vdash f_i(c_{i_1},..., c_{i_k})=c_{i_n}$
  \end{enumerate}

  {\color{red}TUTAJ MAM DOŚĆ NEWELA}
\end{proof}







