\section{22.10.2025}{}

\begin{example}[m]
  \item Jak wygląda zbiór zer wielomianu $y^2-x$ w $\C P^2$? Najpierw musimy ten wielomian ujednorodnić.
    $$\Sigma=\{[x:y:z]\;:\;y^2-xz=0\}$$
    Wstawiając w kolejno $x$, $y$ i $z$ wartość $1$ dostajemy powierzchnie w $\C^2$ (spełnione są założenia twierdzenia o funkcji uwikłanej).
  \item Zera wielomianu $y^2-p(x)$, gdzie $p(x)\in\C[x]$ i $d=\deg(p)>3$.
    Wersja ujednorodniona to
    $$y^{2-d}-\sum_{k=0}^da_kx^kz^{d-k}.$$
    $y^2-p(x)=0$ zadaje w $\C^2$ powierzchnię, o ile $p$ ma jednorodne pierwiastki. 

    Punkty w $\C P^2$ dla których $z=0$ to $-a_dx^d=0$. Jest w $\Sigma\subseteq \C P^2$ jeden taki punkt: $[0:1:0]$.

    W mapie $y=1$ mamy zera wielomianu
    $$q(x,z)=z^{d-2}-\sum a_kx^kz^{d-k}=0$$
    punktowi $[0:1:0]$ w tej mapie odpowiada punkt $(0,0)$. Ale $q_x(0,0)=q_z(0,0)=0$. Dziś będziemy w tym przypadku konstruować nieosobliwą powierzchnię Riemanna.
\end{example}

Niech $p(x,y)\in\C[x,y]$ będzie nierozkładalny (i różny od $ax+b$).
$$p(x,y)=a_0(x)y^n+a_1(x)y^{n-1}+...+a_n(x),$$
gdzie $a_i(x)\in\C[x]$.

Zbiór $S_1=\{x\in \C\;:\;(\exists\;y\in\C)\;p(x,y)=0=p_y(x,y)\}$
jest skończony. Podobnie $S_0=\{x\in\C\;:\;a_0(x)=0\}$. Czyli $S=S_0\cup S_1\cup\{\infty\}$ jest skończonym podzbiorem $\overline{\C}$.
$$pr_1:\C^2\to\C$$
$$Reg:=Z(p)\cap pr_1^{-1}(\overline{\C}-S)$$
będzie obrazek

\begin{fact}{}{}
  $\Pi=pr_1|_{Reg}:Reg\to\overline{\C}-S$ jest $n$-krotnym nakryciem.
\end{fact}

\begin{definition}{}{}
  tutaj można nakrycia definiować
\end{definition}

\begin{example}[m]
  \item $\R^1\to S^1$
  \item $S^1\to S^1$
  \item $\C\to\C/\Z[i]$
  \item $exp:\C\to\C-\{0\}$
  \item $B(0, r^{1/k})-\{0\}\to B(0,r)-\{0\}$ przez $z\mapsto z^k$
\end{example}

\begin{example}
  antyprzykład: $\pi:O_\C\to \C$ jest lokalnym homeomorfizmem

  $\pi^{-1}(B(0,\epsilon))$ ma składową spójną
  $$\{\underline{(\frac{1}{z})}_w\;:\;w\in B(0,\epsilon)-\{0\}\}=U$$
  $\pi|_ULU\to B(0,\epsilon)$ nie jest homeomorfizmem

  rysunek?
\end{example}

\begin{fact}{}{}
  $\Pi:Reg\to \overline{\C}-S$ est $n$-krotnym nakryciem
\end{fact}

\begin{proof}
  Niech $x_0\in\overline{\C}-S$ i produkujemy otoczenie żeby się zgadzało.

  $$\pi^{-1}(x_0)=\{(x_0, y)\;:\;p(x_0,y)=0\},$$
  gdzie $p(x_0, y)$ jest wielomianem zmiennej $y$ stopnia $n$ o pojedynczych pierwiastkach $y_1,..., y_n\in\C$, więc $|\pi^{-1}(x_0)|=n$.

  Wokół każdego $(x_0, y_i)$ zbiór $Reg$ jest wykresem funkcji

  {\color{red}
    oj coś tutaj skipnęłam z $D$ hehe
    jest w zdjęciu
  }

  Niech $x_1\in D$. Wtedy rozwiązaniami równania $p(x_1, y)=0$ są $y_1(x_1), y_2(x_1),..., y_n(x_1)$ są parami różne i jest ich dokładnie $n$, czyli stopień $p(x_1, y)$ - więc to są wszystkie rozwiązania. Stąd $\pi^{-1}(x_1)\subseteq\bigcup\widetilde{D_i}$.
\end{proof}

Niech $s\in S$. Rozważmy $B(s, \epsilon)$ - koło wokół $s$ - rozłączne z $S-\{s\}$. Niech $B^*(s)=B(s, \epsilon)-\{s\}$ (małe koło wokół $s$ nakłóte w $s$). Wtedy $\pi:\pi^{-1}(B*(s))\to B*(s)$ jest nakryciem (być może niespójnym). 

Niech $B_j*(s)$ będzie składową spójną tego przeciwobrazu.
diagram boże jak mi sie dzisiaj nie chce pisac

\begin{lemma}{nakryciowy}{}
  Istnieje homeomorfizm $f_{s,j}$ zamykający diagram

  Ten $f_{s,j}$ jest biholomorfizmem, bo $\pi$, $z\mapsto z^k$ są lokalnymi boholomorfizmami
\end{lemma}

o jesuuu obrazeeeeeeeeeek

Dla każdego $B_j*(s)$ dokładamy punkcik $\hat{s_j}$. $f_{s,j}$ rozszerzamy deklarując $f_{s,j}(\hat{s_j})$ i dostajemy mapę $B_j*(s)\cup\{\hat{s_j})=:B_j(s)\to B(0, \epsilon^{1/k})$.

W ten sposób zbudowaliśmy tzw. powierzchnię Riemanna $\Sigma(p)\supseteq Reg$ funkcji algebracznej określonej przez $p$.

\begin{theorem}{}{}
  \begin{enumerate}
    \item $\Sigma(p)$ jest zwarta
    \item odwzorowanie $Reeg\ni (x,y)\mapsto y$ określa meromorficzną funkcję na $\Sigma(p)$
    \item $Reg$ jest spójny
    \item Niech $Z_{proj}(p)$ - projektywne uzwarcenie, czyli zbiór zer ujednorodnionego $p$

      Wtedy włożenie $Reg\hookrightarrow Z_{proj}(p)$ rozszerza się do holomorficznej surjekcji $\Sigma(p)\to Z_{proj}(p)$.
  \end{enumerate}
\end{theorem}

Tutaj przykład i obrazek

\begin{proof}
  \begin{enumerate}
    \item Pokryjemy $\Sigma(p)$ skończoną liczbą domkniętych dysków. Wokół każdego $s\in S$ wybieramy mały domknięty dysk $D_s$, którego przeciwobraz w $\Sigma(p)$ to skończona suma domkniętych dysków $D_j(s)$ - domknięte otoczenie $\hat{s_j}$.

      Niech $V=\overline{\C}-\bigcup_{s\in S}D(s)$, wtedy $\overline{V}$ jest zwartym podzbiorem $\overline{\C}-S$. Dla każdego $x\in\overline{V}$ wybieramy dysk $D(x)\subseteq\overline{\C}-S$ prawidłowo nakryty
      
      kolejne zdjęcie
  \end{enumerate}
\end{proof}


