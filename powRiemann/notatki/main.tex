%! TeX program = lualatex
\documentclass[twoside, a4paper, 12pt]{book}

\title{Logika R}
\author{\color{subtext1}Weronika Jakimowicz}
\date{Zima 2025/26}

\usepackage[pl, dates]{../../template}

\usepackage{halloweenmath}

\let\oldamalg\amalg
\newcommand{\uu}[1]{\underline{#1}}
\newcommand{\Mm}{\ensuremath{\mathfrak{M}}}
\newcommand{\lr}{\leftrightarrow}

\DeclareMathOperator{\Cn}{Cn}
\DeclareMathOperator{\Th}{Th}

\newtcolorbox{KONWENCJA}{
  title={Konwencja},
  fonttitle = \bfseries,
  coltitle = blue!40!purple!40!black,
  description color = blue!40!purple!40!black, 
  description font = \bfseries\large,
  colbacktitle = white, %green!40!black!75,
  breakable, 
  enhanced,
  attach boxed title to top left = {
    xshift=0.5cm, 
    yshift=-\tcboxedtitleheight/2
  },
  boxed title style = {
    boxrule=0pt,
    colframe=white
  },
  top=3mm,
  bottom=4mm,
  colback=white,
  frame hidden,
  borderline west = {1pt}{0pt}{purple!40!blue!40},
  % borderline north = {1pt}{0pt}{purple!40!blue!40},
  borderline south = {1pt}{0pt}{purple!40!blue!40},
  overlay={%
    \draw[line width=1pt, purple!40!blue!40]
      ([xshift=0]frame.north west) -- % start 25% in
      ([xshift=0.03\linewidth]frame.north west);  % end 25% before the right edge

    \draw[line width=1pt, purple!40!blue!40]
      ([xshift=0.186\linewidth]frame.north west) -- % start 25% in
      ([xshift=0.3\linewidth]frame.north west);  % end 25% before the right edge

    \filldraw[purple!40!blue!40] ([xshift=0.3\linewidth]frame.north west) circle (1pt);
    \filldraw[purple!40!blue!40] ([xshift=0.31\linewidth]frame.north west) circle (0.8pt);
    \filldraw[purple!40!blue!40] ([xshift=0.32\linewidth]frame.north west) circle (0.4pt);
  },
  arc=0mm,
  after skip=5mm,
  before skip=5mm,
  width=0.9\textwidth
}

\newenvironment{konwencja}{
    \begin{center}
    \begin{KONWENCJA}{}{}
  }{
    \end{KONWENCJA}
    \end{center}
  }

\begin{document}
\frontmatter 
\maketitle
\thispagestyle{empty}
\setcounter{page}{0}

\tableofcontents
\mainmatter

\pagestyle{fancy}

\chapter{Formalizacja matematyki}

\section{Uproszczony model rzeczywistości matematycznej: struktura I rzędu}

\subsection{Model języka i język struktury modelu}

\begin{definition}{model}{}
  \buff{Model} to struktura matematyczna składająca się z
  \begin{itemize}
    \item niepustego zbioru będącego \acc{uniwersum} (dziedzina) $A\neq\emptyset$,
    \item \acc{funkcji} $f_1,..., f_k$ o arności $n_i$ (tzn. $f_i:A^{n_i}\to A$),
    \item \acc{relacji} (orzeczników, predykatów) w $A$, $P_1,..., P_n$, gdzie $P_i\subseteq A^{n_i}$,
    \item \acc{stałych} z $A$ $c_1,..., c_l\in A$.
  \end{itemize}
\end{definition}

Zapisujemy
$$\Mm=(A; f_1,...,f_k; P_1,...,P_n; c_1,...,c_l)$$
gdzie $k,n,l$ to liczby kardynalne, zazwyczaj skończone (tzn. $k,n,l\in\N$).

\begin{konwencja}{}{}
  $$(\Mm:=)|\Mm|:=A$$
\end{konwencja}

\begin{example}[m]
\item Jeśli $n=0$, czyli nie mamy relacji, to \Mm{} jest \acc{strukturą algebraiczną} (algebrą ogólną). Weźmy na przykład grupę. Grupa jest zdefiniowana jako zbiór $G$ z wyróżnionym elementem neutralnym $e$, operacją mnożenia $\cdot$ oraz brania elementu odwrotnego $^{-1}$. Operacje to funkcje, a element neutralny to stała. Sam zbiór $G$ to z kolei uniwersum, czyli mamy model:
  $$(G; \cdot, ^{-1};\;;e)$$
\item Rodzina zbiorów $V$ z relacją należenia $\in$ jest modelem z jedną relacją, ale bez funkcji i bez stałych:
  $$(V;\;;\in;\;)$$
  Taką strukturę nazywamy \acc{strukturą relacyjną}.
\end{example}

Symbole oznaczające funkcje, relacje, stałe będziemy od ich znaczenia odróżniać przez podkreślenie:
\begin{itemize}
  \item $\underline{f}_i$, $\underline{P}_j$, $\underline{c}_t$ to symbole, 
  \item natomiast $f_i$, $P_J$, $c_t$ to funkcja, relacja, stała.
\end{itemize}

\begin{definition}{język}{}
  Język
  $$L=\{\underline{f}_1,...,\underline{f}_k;\underline{P}_1,..., \underline{P}_n; \underline{c}_1,...,\uu{c}_l\}$$
  składa się z symboli: funkcyjnych, relacyjnych, stałych wraz z przypisanymi tym symbolom arnościami, tzn. $f_j$ to symbol funkcji $n_i$-argumentowej etc.
\end{definition}

Język jak wyżej jest nazywany językiem struktury \Mm{}, typem podobieństwa \Mm{}, \buff{sygnaturą \Mm{}}. Z kolei \Mm{} jest modelem dla $L$.

\begin{konwencja}{}{}
Szerzej będziemy dla \Mm{} - modelu dla $L$ - pisać
$$(\Mm; \uu{f_1}^\Mm, ..., \uu{f_k}^\Mm; \uu{P_1}^\Mm,...,\uu{P_n}^\Mm; \uu{c_1}^\Mm,...,\uu{c_l}^\Mm)$$
gdzie $\uu{f_i}^\Mm$ oznacza interpretację symbolu $\uu{f_1}^\Mm$ w kontekście modelu \Mm.
\end{konwencja}

\begin{remark}{}{}
  Dla dowolnego języka $L$ istnieje wiele struktur \Mm.
\end{remark}

\begin{example}
  Język grup
\end{example}

Mając dany język $L$ mówimy/piszemy w nim przy pomocy
\begin{itemize}
  \item symbolów języka,
  \item symboli logicznych $\land$, $\lor$, $\neg$, $\to$, $\lr$ (!!! $\implies$ oraz $\iff$ będą dla nas elementami metajęzyka !!!), $\forall$, $\exists$, $=$,
  \item zmiennych, np. $x_i$ dla $i\in \N$, $y$, $z$,
  \item oraz symboli pomocniczych takich jak nawiasy, przecinki etc.
\end{itemize}

\begin{remark}{}{}
  Spójniki można ograniczyć do $\land$, $\neg$ i kwantyfikatora $\exists$. Całą resztę spójników można zdefiniować jako macra przy pomocy tych trzech, np.
  $$p\lor q\quad :=\quad \neg(\neg p\land \neq q)$$
\end{remark}

\textbf{Wyrażenia języka $L$:}
\begin{enumerate}[label=\alph*)]
  \item \buff{wyrażenia nazwowe} (termy) należą do $\mathcal{T}_L$ ($\text{Term}_L$) i są definiowane rekurencyjnie:
    \begin{itemize}
      \item zmienna, symbol stałej należą do $\mathcal{T}_L$ i nazywają się \acc{termami atomowymi}
      \item jeśli $\tau_1$, ..., $\tau_n\in \mathcal{T}_L$, a $\uu{f}$ jest symbolem $n$-argumentowej funkcji z $L$, to $\uu{f}(\tau_1,...,\tau_n)\in \mathcal{T}_L$ i nazywa się \acc{termem złożonym}. 
    \end{itemize}
  \item \buff{formuły} oznaczamy $\mathcal{F}_L$ ($\text{Form}_L$) i definiujemy rekurencyjnie w następujący sposób
    \begin{itemize}
      \item dla wszystkich termów $\tau_1,...,\tau_n$ zachodzi $(\tau_1=\tau_2)\in \mathcal{F}_L$ oraz dla $n$-argumentowego symbolu relacji $\uu{P}_j$: $\uu{P}_j(\tau_1,...,\tau_n)\in  \mathcal{F}_L$ - to są \acc{formuły atomowe},
      \item $\phi,\psi\in\mathcal{F}_L\implies (\neg \phi), (\phi\land\psi)\in \mathcal{F}_L$

        $\phi\in \mathcal{F}_L$ $\implies (\exists\;r\;\phi),(\forall\;r\phi)\in\mathcal{F}_L$ ($r$ występujące w wyrażeniach nazywamy zmiennymi) 
        
        Formuły z tego punktu nazywamy \acc{formuły złożone}.
    \end{itemize}
\end{enumerate}

\buff{Hierarchia symboli} logicznych umożliwia pomijanie nawiasów:
\begin{enumerate}
  \item symbole matematyczne
  \item kwantyfikatory
  \item negacja $\neg$
  \item $\land$, $\lor$
  \item $\to$, $\lr$
\end{enumerate}

\subsection{Zdania w języku}

Niech $\phi\in \mathcal{F}_L$ będzie formułą w której występuje, co najmniej raz, zmienna $v$. Jeśli pewne wystąpienie $v$ w $\phi$ jest w zasięgu pewnego kwantyfikatora $Q_v\in\{\forall,\;\exists\}$, to spośród wszystkich wystąpień $Q_v$ w $\phi$ w których zasięgu jest $v$ wybieramy to najbardziej na prawo i mówimy, że to \buff{$Q_v$ wiąże dane wystąpienie $v$} w $\phi$. Na przykład
\begin{center}
  \begin{tikzpicture}
    \node at (0,0) {$\forall\;x\;\exists\;y\;(x\in y\;\land\;\forall\;x\;(x\in y \to x=y))$};
    \draw[->] (-2.95, -.1) -- (-2.95, -.5) -- (2.4, -.5) -- (2.3, -.1);
    \draw[->, green] (.2, -.1) -- (.2, -.7) -- node[midway, below] {wiąże} (2.1, -.7) -- (2.2, -.1);
  \end{tikzpicture}
\end{center}

Jeśli nie ma kwantyfikatora $Q_v$ jak wyżej, to wystąpienie $v$ w $\phi$ jest \buff{wolne}. Popatrzmy na przykład na formułę
$$\exists {\color{red}y}\;x\in {\color{red}y}\;\land\;\forall\;x\;(x\in {\color{blue}y}\to x={\color{blue}y}).$$
Kwantyfikator z czerwonym $y$ wiąże czerwony $y$, a niebieskie $y$ pozostają wolne. Dodając nawias pierwszy kwantyfikator wiąże już wszystkie występienia $y$:
$$\exists {\color{red}y}\;(x\in {\color{red}y}\;\land\;\forall\;x\;(x\in {\color{red}y}\to x={\color{red}y})).$$

\begin{konwencja}{}{}
  Zapis $\phi(v_1,...,v_n)$ oznacza, że wszystkie wolne zmienne w $\phi$ są wśród $v_1$, ..., $v_n$ (ale niekoniecznie wszystkie $v_i$ są zmiennymi wolnymi).
\end{konwencja}

\begin{definition}{zdanie}{}
  Formalne zdanie w języku $L$ to formuła niezawierająca zmiennych wolnych.
\end{definition}

Tarski podał na początku XX wieku definicję prawdy. Nieco później udowodnił twierdzenie, że nie da się prawdy zdefiniować.

\buff{Powstaje pytanie co to znaczy, że formuła z $L$ jest prawdziwa w strukturze \Mm{} dla $L$?}

Niech \Mm{} będzie modelem dla $L=\{\uu{f}_i,...,\uu{P}_j,...,\uu{c}_t,...\}$ oraz $\{\uu{a}\;:\;a\in \Mm\}$ niech będzie zbiorem nowych symboli stałych. Rozważmy nowy, większy język 
$$L(\Mm)=L\cup \{\uu{a}\;:\;a\in |\Mm|\},$$
którego modelem nadal jest \Mm{} ($\uu{a}^\Mm:=a$).

Termy stałe $\sigma^\Mm\in |\Mm|$ z $L(\Mm)$ interpretujemy w \Mm{} w następujący sposób:
\begin{itemize}
  \item jeśli $\uu{\tau}$ jest symbolem stałym $\uu{c}_i$ w $L$, to $\tau^\Mm\in\Mm$ jest interpretacją $\uu{c}_i^\Mm$
  \item jeśli $\tau=\uu{f}_i(\uu{\tau}_1,...,\uu{\tau}_n)$ jest termem złożonym, to $\sigma^\Mm=\uu{f}_i^\Mm(\uu{\tau}_1^\Mm,...,\uu\tau_n^\Mm)$. 
\end{itemize}

\begin{konwencja}{}{}
  $\Mm\models\phi$ oznacza, że $\phi$ jest prawdziwe/spełniane w \Mm.
\end{konwencja}

\begin{enumerate}[label=\alph*)]
  \item zdania atomowe:
    \begin{itemize}
      \item $\Mm\models \tau_1=\tau_2\iff\tau_1^\Mm=\tau_2^\Mm$
      \item $\Mm\models \uu{P}_j(\tau_1,..., \tau_n)\iff (\tau_1^\Mm,...,\tau_n^\Mm)\in\uu{P}_j^\Mm$
    \end{itemize}
  \item zdania złożone:
    \begin{itemize}
      \item $\Mm\models\phi\;\land\;\phi\iff\Mm\models\phi\text{ oraz }\Mm\models\psi$
      \item $\Mm\models\neg\phi\iff $ nieprawda, że $\Mm\models \phi$ (oznaczamy $\Mm\not\models\phi$)
      \item $\Mm\models\exists\;v\;\phi$ (jest tym samym co $\exists\;v\;\phi(v)$, bo zakładamy, że nie ma innych zmiennych wolnych w $\phi$, bo to jest zdanie) $\iff \Mm\models\underset{=\phi(\uu{a})}{\phi(v/\uu{a})}$ dla pewnego $a\in|\Mm|$, gdzie $\phi(\uu{a})$ jest formułą powstającą z $\phi$ przez zastąpnienie każdego wolnego wystąpienia $v$ w $\phi$ przez $\uu{a}$.
    \end{itemize}
\end{enumerate}

To daje nam, że $\Mm\models\phi$ dla każdego $\phi\in\mathcal{F}_L$.

Kwantyfikatujemy po zmiennych, nie po podzbiorach uniwersum, bo to jest logika I rzędu. Zbiory tworzymy w umyśle, postrzegamy indywidua. Teorie wyższych rzędów nie są absolutne.

% \buff{Atomizm logiczny} - każde zdanie można zredukować do zdań atomowych, których już dalej się rozbić nie da. 

% Można tutaj od razu udowodnić, że zachodzi prawo wyłączonego środka.

% Spełnianie dla formuł $\phi(v_1,...,v_n)$ języka $L$:
% \begin{itemize} $(a_1,...,a_n)\subseteq |\Mm| $ spełnia $\phi$ jak wyżej w $\Mm$, jeśli $\Mm\models\phi(\uu{a}_1,...,\uu{a}_n)$
%   \item uniwersalne domknięcie $\phi(v_1,...,v_n)$ to formuła
%     $$\overline{\phi}=\forall\;v_1,...,\forall\;v_n\;\phi(v_1,...,v_n)$$
%     z dokładnością do równoważności formuł jest jednoznacznie zdefiniowane $\Mm\models\phi\iff\Mm\models\overline\phi$
% \end{itemize}

\begin{definition}{spełnianie formuły}{}
  Krotka $\langle a_1,...,a_n\rangle$ elementów z uniwersum $\Mm$ \buff{spełnia formułę} $\phi$ języka, gdy $\Mm\models\phi(\uu{a}_1,...,\uu{a}_n)$.
\end{definition}

\begin{definition}{uniwersalne domknięcie}{}
  Dla formuły $\phi$ języka $L$ jej \buff{uniwersalne domknięcie} $\phi$ to formuła
  $$\overline{\phi}:=\forall\;x_1\;\forall\;x_2\;...\forall\;x_n\;\phi$$
\end{definition}
W następnej części wykładu pokażemy, że 
$$\Mm\models\phi\iff\Mm\models\overline{\phi}$$

Powyższe zagadnienia mówią, że domyślny kwantyfikator to kwantyfikator $\forall$. To ma spełnienie w naturalnej matematyce, np. przemienność dodawania $(\R, +)\models\;x+y=y+x$ gdzie pomijamy $\forall\;x\;\forall\;y$.

\subsection{Tautologie}

\begin{definition}{tautologia KRL}{}
  Niech $\phi\in\mathcal{F}_L$. Wtedy $\phi$ jest \buff{tautologią} klasycznego rachunku logicznego [KRL], gdy jest zawsze prawdziwe. Zapisujemy to $\models\phi$, co oznacza $\forall\;\Mm$ modelu $L$ $\Mm\models\phi$.
\end{definition}

\buff{Jak rozpoznać, czy $\models\phi$?} Ogólnie jest to pytanie nierozstrzygalne (twierdzenie G\"odla).

Niektóre tautologie łatwo rozpoznać, jak przykłady niżej.
\begin{example}[m]
  \item $x=x$
  \item $=$ jest relacją równoważności
\end{example}

Przygodę z tautologiami rozpoczynamy od \buff{tautologicznych formuł zdaniowych} (czyli schematów).

Niech $Z=\{p_0, p_1,...,p_n, r, q,...\}$ będzie zbiorem zmiennych zdaniowych. Zbiór formuł zdaniowych $S=S_Z$ nad $Z$ definiujemy w następujący sposób:
\begin{itemize}
  \item $v\in Z\implies v\in S$
  \item $\alpha, \beta\in S\implies \neg\alpha, \alpha\;\land\;\beta\in S$
  \item skrótowo: $\lor, \to, \lr\in S$
\end{itemize}

\begin{definition}{wartościowanie}{}
  \buff{Wartościowanie logiczne formuł zdaniowych} to dowolna funkcja $v:S\to \{0,1\}$ taka, że dla każdych $\alpha,\beta\in S$ 
  \begin{itemize}
    \item $v(\neg\alpha)=1-v(\alpha)$
    \item $v(\alpha\;\land\;\beta)=\min\{v(\alpha), v(\beta)\}$
  \end{itemize}
\end{definition}

\begin{definition}{tautologia KRZ}{}
  Powiemy, że $\alpha\in S$ jest \buff{tautologią} ($\models\alpha$) gdy dla każdego wartościowania $v$ zachodzi $v(\alpha)=1$.
\end{definition}

Na przykład: $\models\neg(\alpha\;\land\;\neg\alpha)$ dla każdego $\alpha\in S$

Istnieje algorytm rozstrzygający, czy $\models \alpha$ dla $\alpha\in S$ (metoda 0-1): wartość $v(\alpha)$ zależy tylko od $v(x)$ dla zmiennych zdaniowych $x$ w $\alpha$.

\begin{definition}{przykład formuły}{}
  Załóżmy, że $\alpha\in S$ jest zbudowany ze zmiennych $p_1,..., p_n\in Z$ oraz $\phi_1,..., \phi_n\in\mathcal{F}_L$ i $\phi\in\mathcal{F}_L$ powstaje z $\alpha$ przez zastąpienie $p_i$ przez $\phi_i$ wszędzie w $\alpha$ (zapisujemy $\phi=\alpha(p_1/\phi_1,...,p_n/\phi_n)$. Mówimy, że $\phi$ jest \buff{jest przykładem formuły $\alpha$}. 
\end{definition}

Weźmy na przykład formułą $\alpha=p_1\;\land\;p_2$, wtedy $\phi=\phi_1\;\land\;\phi_2$ jest przykładem formuły $\alpha$.

\begin{theorem}{}{}
  Jeśli $\models\alpha$ i $\phi$ jest przykładem $\alpha$, to $\models\phi$.
\end{theorem}
\begin{proof}
  Ćwiczenie
\end{proof}

\begin{definition}{reguła wnioskowania}{}
  Reguła wnioskowania składa się z:
  \begin{itemize}
    \item przesłanek $\phi_1,....,\phi_n\in\mathcal{F}_L$
    \item tezy $\phi$.
  \end{itemize}
  Zapisujemy 
  $$\frac{\phi_1,...,\phi_n}{\phi}$$

  Reguła wnioskowania jest poprawna, kiedy nie prowadzi nas na manowce. 
  \begin{enumerate}
    \item $\frac{\phi_1,...,\phi_n}{\phi}$ jest \buff{poprawna}, jeśli dla każdego modelu $\Mm$ dla $L$ 
      $$\Mm\models\phi_1\;\land...\land\;\phi_n\implies \Mm\models\phi$$
    \item $\frac{\alpha_1,...,\alpha_n}{\alpha}$ jest poprawna, jeśli dla każdego wartościowania $v$ 
      $$v(\alpha_1)=...=v(\alpha_n)=1\implies v(\alpha)=1$$
  \end{enumerate}
\end{definition}

\begin{example}[m]
  \item Modus Ponens (reguła odrywania, cut rule) 
    $$\frac{\alpha,\alpha\to \beta}{\beta}$$
  \item reguła generalizacji ($\forall\;$-rule)
    $$\frac{\phi}{\forall\;v\;\phi(v)}$$
\end{example}

% KRL - $\phi$ etc, KRZ - $\alpha$, $\beta$ etc

\section{Aksjomatyczne ujęcie KRL}
\begin{enumerate}[label=\textbf{\color{green}(A\arabic*)},ref=\color{green!60!black}(A\arabic*), start=0]
  \item\label{A0} formuła będąca przykładem dowolnego zdania KRZ, które jest tautologią
  \item\label{A1} $\forall\;v\;(\phi\to\psi)\to(\phi\to\forall\;v\;\psi)$, gdy $v$ nie jest wolna w $\phi$
  \item\label{A2} $\forall\;v\;\phi\to\phi(v/t)$ (to jest poprawny zapis, ale dla pewności: $(\forall\;v\;\phi)\to\phi(v/t)$)

    $t$ to jest term podstawiany za każde wolne wystąpienie $v$ w $\phi$ pod warunkiem, że żadne z takich wystąpień nie jest w zasięgu kwantyfikatora wiążącego zmienną występującą w termie $t$.
\end{enumerate}

Zastrzeżenie aksjomacie \ref{A2} jest istotne: jeśli weźmiemy jako $\phi$ formułę 
$$\phi = \exists\;y\;x\neq y$$ 
i podstawimy za wolne wystąpienia $x$ term $t=y$, to 
$$\phi(x/t)=\exists\;y\;y\neq y$$ 
co z \ref{A2} dałoby
$$\not\models\forall\;x\phi\to\phi(x/t)=\forall\;\exists\;y\;x\neq y\to \exists\;y\;y\neq y$$

\begin{definition}{aksjomaty równości}{}
{Aksjomaty równości} ($v_i$ - wolne zmienne):
\begin{enumerate}[label=\textbf{\color{orange}(R\arabic*)},ref=\color{orange!60!black}R\arabic*]
  \item\label{R1} $v_1=v_1$
  \item\label{R2} $v_1=v_2\to t(...v_1...)=t(...v_1/v_2...)$ 
  \item\label{R3} $v_1=v_2\to (\phi(...v_1...)\to \phi(...v_2...))$ gdzie $v_1$ to wolne wystąpienie w $\phi$ niebędące w zasięgu kwantyfikatora wiążącego $v_2$.
\end{enumerate}
\end{definition}

\subsection{Dowodliwość}
 
\begin{definition}{dowód formalny}{}
  Niech $X\subseteq\mathcal{F}_L$ będzie dowolnym podzbiorem formuł. 

  Powiemy, że \buff{$X$ dowodzi $\phi\in \mathcal{F}_L$}, 
  $X\vdash \phi$ wtedy i tylko wtedy gdy istnieje ciąg formuł \buff{(dowód formalny)} $\alpha_1$, .., $\alpha_n=\phi$ takie, że dla każdego $i\leq n$
  \begin{enumerate}
    \item $\alpha_i\in X$ lub $\alpha_i$ jest aksjomatem KRL
    \item $\alpha_i$ wynika z $\{\alpha_1,...,\alpha_{i-1}\}$ na mocy 
      \begin{itemize}
        \item Modus Ponens, tzn. $(\exists\;j,t<i)\;\alpha_t=(\alpha_j\to \alpha_i)$
          $$\frac{\alpha_j,\alpha_j\to\alpha_i}{\alpha_i}$$
        \item lub $\forall$-rule $(\exists\;j<i)\;\alpha_i=(\forall\;v\;\alpha_j)$
          $$\frac{\alpha_j}{\forall\;v\;\alpha_j}$$
      \end{itemize}
  \end{enumerate}
\end{definition}

\begin{konwencja}{}{}
  $\vdash \phi$ gdy $\emptyset\vdash\phi$
\end{konwencja}

\begin{definition}{teza KRL}{}
  Jeśli $\phi$ jest takie, że $\vdash\phi$, to $\phi$ nazywamy \buff{tezą} KRL.
\end{definition}

\begin{remark}{}{skonczony dowodzi}
  $X\vdash \phi$ $\iff$ $\exists\;X_0\subseteq X$ skończony $X_0\vdash \phi$
\end{remark}

\begin{example}
  Pokażemy, że $\vdash \forall\;x\;\phi\to\exists\;x\;\phi$, co jest równoważne
  $$\vdash \forall\;x\;\phi\to \neg\;\forall\;x\;\neg\phi$$
  
  \begin{enumerate}
    \item $\alpha_1:\forall\;x\;\phi\to \phi(x/y)$ (A2)
    \item $\alpha_2:\forall\;x\;\neg\phi\to\neg\phi(x/y)$ (A2)
    \item $\alpha_3:\alpha_2\to (\phi(x/y)\to \neg\forall\;x\;\neg\phi)$ (A0 $(p\to\neg q)\to(q\to\neg p)$)
    \item $\alpha_4:\phi(x/y)\to\neg\forall\;x\;\neg\phi$ ($\alpha_2,\alpha_3$ oraz Modus Ponens)
    \item $\alpha_5:\alpha_1\to(\alpha_4\to (\forall\;x\;\phi\to\neg\forall\;x\;\neg\phi))$ (A0 $(p\to q)\to((q\to r)\to(p\to r))$)
    \item $\alpha_6:(\forall\;x)\;\phi\to\neg(\forall\;x)\neg\phi$ (Modus Ponens)
  \end{enumerate}
\end{example}

Rozważmy $\phi$ dane $x=x$. Zdanie
$$\vdash(\forall\;x)\;x=x\to(\exists\;x)x=x$$
jest fałszywe w dziedzinie putestj, czyli modelu
$$(\emptyset; f_1,...,f_n; P_1,...,P_j;)$$
gdzie nie ma elementów uniwersum. Czyli coś z aksjomatów musi zawodzić w dziedzinie pustej - pytanie co?


\begin{center}
  \buff{Od tej pory modele są niepuste}
\end{center}

\begin{remark}{}{}
  Poprawność (soundness) KRL
  $$\vdash\phi\implies \models\phi$$
\end{remark}
\begin{proof}
  Dowód polega na indukcji względem długości formuły (albo długości dowodu) i jest pozostawiony jako ćwiczenie.
\end{proof}

\begin{theorem}{G\"odela o pełności KRL}{}
  $$\models\phi\implies\vdash\phi$$
\end{theorem}

Czyli $\models\phi\iff\vdash\phi$.

\begin{definition}{zbiór konsekwencji}{}
  Dla $X\subseteq\mathcal{F}_L$ \buff{zbiór konsekwencji} $X$ to $\Cn(X)=\{\phi\in\mathcal{F}_L\;:\;X\vdash \phi\}$
  \begin{itemize}
    \item $X$ jest teorią, gdy $X=\Cn(X)$
    \item $A$ jest zbiorem aksjomatów teorii $X$, gdy $X=\Cn(A)$
    \item \acc{zbiory $X$, $Y$ są równoważne}, gdy $\Cn(X)=\Cn(Y)$ co równoważnie można powiedzieć: $X\vdash Y$ i $Y\vdash X$
  \end{itemize}
\end{definition}

\begin{theorem}{twierdzenie o dedukcji}{tw dedukcja}
  Jeśli $\phi$ jest zdaniem, to $(X\vdash\phi\to\psi\iff X\cup\{\phi\}\vdash\phi)$
\end{theorem}

\begin{proof}
  Dowód powyższego twierdzenia jest zadaniem na 1 liście zadań.
\end{proof}

\begin{fact}{}{}
  $\phi$ oraz $\overline{\phi}$ są równoważne, czyli $\phi\vdash\overline{\phi}$ oraz $\overline{\phi}\vdash\phi$.
\end{fact}
\begin{proof}
  $\phi\vdash\overline{\phi}$ 

  Wystarczy zastosować $\forall$-regułę tyle razy ile dopisaliśmy kwantyfikatorów w $\overline{\phi}$.

  $\overline{\phi}\vdash\phi$ 

  Aksjomat \ref{A2} mówi, że $\overline{\phi}\to\phi$. Używając do tego \hyperref[twierdzenia o dedukcji]{th:tw dedukcja} dostajemy
  $$\emptyset\cup\{\overline{\phi}\}\vdash \phi\iff\emptyset\vdash\overline{\phi}\to\phi.$$
\end{proof}

\begin{conclusion}{}{}
  $X\subseteq\mathcal{F}_L$

  $X$ oraz $\{\overline{\phi}\;:\;\phi\in X\}$ są równoważne
\end{conclusion}

\begin{definition}{zbiór sprzeczny}{}
  $X$ jest \buff{sprzeczny}, jeżeli $X\vdash\phi$ oraz $X\vdash\neg\phi$ dla pewnego zdania $\phi\in\mathcal{F}_L$
\end{definition}

W przeciwnym razie mówimy, że $X$ jest niesprzeczny.

\begin{definition}{zbiór zupełny, rozstrzygalny}{}
  \begin{enumerate}
    \item Powiemy, że \buff{\Mm{} jest modelem $X$}, oznaczane $\Mm\models X\iff (\forall\;\phi\in X)\;\Mm\models\phi$.
    \item $X$ jest \buff{zupełny} $\iff (\forall\;\text{zdania }\phi)\;(X\vdash \phi\text{ lub } X\vdash \neg\phi$.
    \item $X$ jest \buff{rozstrzygalny} $\iff$ istnieje algorytm rozstrzygający, czy $X\vdash\phi$. 
  \end{enumerate}
\end{definition}

\begin{example}[m]
\item \acc{Teoria struktury \Mm{}}, definiowana jako zbiór $\Th(\Mm):=\{\phi\;:\;M\models\phi\}$, jest niesprzeczna i zupełna.
\item $\emptyset$ jest niesprzeczny
\end{example}

\begin{theorem}{G\"odela o istnieniu modelu}{}
  Jeśli $S$ jest niesprzecznym zbiorem zdań (równoważnie: formuł), to $S$ ma model.
\end{theorem}

\begin{proof}
  Przedstawimy dowód Leona Henkina dla przypadku, gdy $L$ i $S$ są przeliczalne. Ogólny dowód jest analogiczny, ale wymaga nieprzyjemnych fikołków kombinatorycznych na liczbach kardynalnych.

  Zacznijmy od powiększenia języka $L$ o nowe symbole
  $$L'=L\cup\{c_n\;:\;n<\omega\}$$
  i ponumerowania jego formuł postaci $\phi(x)$ (z co najwyżej jedną wolną zmienną $x$)
  $$\{\phi_n\;:\;n<\omega\}$$

  Zdefiniujmy pomocniczą rosnącą funkcję $f:\omega\to\omega$ taką, że $c_{f(n)}$ nie występuje w formułach $\phi_0(x)$, $\phi_1(x)$, ..., $\phi_n(x)$. Przy pomocy $f$ definiujemy rodzinę zbiorów
  $$S_n:=S\cup\{\underbrace{(\exists\;x)\;\phi_i(x)\to\phi_i(c_{f(i)})}_{\text{aksjomat Henkina}}\;:\;i<n\}.$$
  Mamy $S=S_0\subseteq S_1\subseteq ...$ i niech
  $$S_\omega=\bigcup S_n$$

  \begin{fact}{}{}
    Zbiór $S_\omega$ jest niesprzeczny.
  \end{fact}

  \begin{proof}
    Załóżmy nie wprost, że $S_\omega$ jest sprzeczny. Sprzeczność ta wynika ze skończonego podzbioru (patrz \ref{rm:skonczony dowodzi}), czyli istnieje $n$ takie, że $S_{n+1}$ jest sprzeczny. Wybierzmy najmniejsze takie $n$. Dla pewnego zdania $\alpha$ mamy
    $$S_{n+1}\vdash\alpha\;\land\;\neg\alpha.$$
    Zapiszmy $S_{n+1}$ jako
    $$S_{n+1}:=S_n\cup\{\underbrace{(\exists\;x)\;\phi_n(x)\to\phi_n(c_{f(i)})}_{H}\},$$
    w tedy z twierdzenia o dedukcji (\ref{th:tw dedukcja}) mamy
    $$S_n\cup\{H\}\vdash \alpha\;\land\;\neg\alpha\quad\iff\quad S_n\vdash H\to\alpha\;\land\;\neg\alpha$$
    Korzystając z aksjomatu \ref{A0} dla $(p\to(q\;\land\;\neg q))\to\neg p$ i Modus Ponens dostajemy
    $$S_n\vdash \neg((\exists\;x)\;\phi_n(x)\to\phi_n(c_{f(n)}))$$
    korzystając jeszcze raz z \ref{A0} ($\neg(p\to q)\to (p\;\land\;\neg q)$) oraz MP mamy
    $$S_n\vdash (\exists\;x)\;\phi_n(x)\;\land\;\neg\pni_n(c_{f(n)})$$
    z czego wnioskujemy, że
    \begin{align*}
      &S_n\vdash (\exists\;x)\;\phi_n(x)\\ 
      &S_n\vdash \neg\phi_n(c_{f(n)})
    \end{align*}
    Wybieraliśmy funkcję $f$ tak, żeby symbol $c_{f(n)}$ nie występował w zdaniach $S_n$, więc możemy go zamienić na dowolny inny, np $y$, a następnie skorzystać z $\forall$-reguły by dostać
    $$S_n\vdash \neg\phi_n(y)\implies S_n\vdash (\forall\;y)\;\neg\phi_n(y).$$
    Dzięki \ref{A2} i faktowi, że $c_{f(n)}$ nie występowało nigdzie indziej w $\phi_n$ (w tym nie było nigdy wiązane przez kwantyfikatory), możemy zamienić
    $$S_n\vdash (\forall\;y)\;\neg\phi_n(y)\to \neg\phi_n(y/x).$$
    Korzystając z MP mamy
    $$S_n\vdash \neg\phi_n(x),$$
    a z $\forall$-reguły:
    $$S_n\vdash (\forall\;x)\;\neg\phi_n(x)$$
    co pokazuje, że $S_n$ jest sprzeczne, bo $S_n\vdash (\exists\;x)\;\phi_n(x)$.
  \end{proof}






  Twierdzenie Lidenbauma(???) jeśli mamy niesprzeczny zbiór formuł/zdań to możemy go rozszerzyć do maksymalnego niesprzecznego zbioru formuł/zdań, który jest zupełny i niesprzeczny zbiór formuł/zdań
  
  Korzystamy, żeby dostać $S'\supseteq S_\omega$, który jest zupełny i niesprzecznym zbiorem zdań w $L'$

  $S'$ opisuje strukturę modelu na $\{c_n\;:\;n<\omega\}$

  mamy relację równoważności na zbiorze $C=\{c_n\;:\;n<\omega\}$: $c_n\sim c_m\iff S'\vdash c_n=c_m$
  \begin{itemize}
    \item symetria: aksjomat równości
  \end{itemize}

  Niech $\Mm=\{c_n/\sim\;:\;n\in\omega\}=C/\sim$
  $L'$ - struktura na $M$
  \begin{enumerate}
    \item $P_i^\Mm(c_{i_1}/\sim,...,c_{i_n}/\sim)\iff S'\vdash P_i(c_{i_1},...,c_{i_n})$
    \item $f^\Mm_i(c_{i_1}/\sim,...,c_{i_k}/\sim)=c_{i_n}/\sim\iff S'\vdash f_i(c_{i_1},..., c_{i_k})=c_{i_n}$
  \end{enumerate}

  {\color{red}TUTAJ MAM DOŚĆ NEWELA}
\end{proof}









% \section{22.10.2025}{}

\begin{example}[m]
  \item Jak wygląda zbiór zer wielomianu $y^2-x$ w $\C P^2$? Najpierw musimy ten wielomian ujednorodnić.
    $$\Sigma=\{[x:y:z]\;:\;y^2-xz=0\}$$
    Wstawiając w kolejno $x$, $y$ i $z$ wartość $1$ dostajemy powierzchnie w $\C^2$ (spełnione są założenia twierdzenia o funkcji uwikłanej).
  \item Zera wielomianu $y^2-p(x)$, gdzie $p(x)\in\C[x]$ i $d=\deg(p)>3$.
    Wersja ujednorodniona to
    $$y^{2-d}-\sum_{k=0}^da_kx^kz^{d-k}.$$
    $y^2-p(x)=0$ zadaje w $\C^2$ powierzchnię, o ile $p$ ma jednorodne pierwiastki. 

    Punkty w $\C P^2$ dla których $z=0$ to $-a_dx^d=0$. Jest w $\Sigma\subseteq \C P^2$ jeden taki punkt: $[0:1:0]$.

    W mapie $y=1$ mamy zera wielomianu
    $$q(x,z)=z^{d-2}-\sum a_kx^kz^{d-k}=0$$
    punktowi $[0:1:0]$ w tej mapie odpowiada punkt $(0,0)$. Ale $q_x(0,0)=q_z(0,0)=0$. Dziś będziemy w tym przypadku konstruować nieosobliwą powierzchnię Riemanna.
\end{example}

Niech $p(x,y)\in\C[x,y]$ będzie nierozkładalny (i różny od $ax+b$).
$$p(x,y)=a_0(x)y^n+a_1(x)y^{n-1}+...+a_n(x),$$
gdzie $a_i(x)\in\C[x]$.

Zbiór $S_1=\{x\in \C\;:\;(\exists\;y\in\C)\;p(x,y)=0=p_y(x,y)\}$
jest skończony. Podobnie $S_0=\{x\in\C\;:\;a_0(x)=0\}$. Czyli $S=S_0\cup S_1\cup\{\infty\}$ jest skończonym podzbiorem $\overline{\C}$.
$$pr_1:\C^2\to\C$$
$$Reg:=Z(p)\cap pr_1^{-1}(\overline{\C}-S)$$
będzie obrazek

\begin{fact}{}{}
  $\Pi=pr_1|_{Reg}:Reg\to\overline{\C}-S$ jest $n$-krotnym nakryciem.
\end{fact}

\begin{definition}{}{}
  tutaj można nakrycia definiować
\end{definition}

\begin{example}[m]
  \item $\R^1\to S^1$
  \item $S^1\to S^1$
  \item $\C\to\C/\Z[i]$
  \item $exp:\C\to\C-\{0\}$
  \item $B(0, r^{1/k})-\{0\}\to B(0,r)-\{0\}$ przez $z\mapsto z^k$
\end{example}

\begin{example}
  antyprzykład: $\pi:O_\C\to \C$ jest lokalnym homeomorfizmem

  $\pi^{-1}(B(0,\epsilon))$ ma składową spójną
  $$\{\underline{(\frac{1}{z})}_w\;:\;w\in B(0,\epsilon)-\{0\}\}=U$$
  $\pi|_ULU\to B(0,\epsilon)$ nie jest homeomorfizmem

  rysunek?
\end{example}

\begin{fact}{}{}
  $\Pi:Reg\to \overline{\C}-S$ est $n$-krotnym nakryciem
\end{fact}

\begin{proof}
  Niech $x_0\in\overline{\C}-S$ i produkujemy otoczenie żeby się zgadzało.

  $$\pi^{-1}(x_0)=\{(x_0, y)\;:\;p(x_0,y)=0\},$$
  gdzie $p(x_0, y)$ jest wielomianem zmiennej $y$ stopnia $n$ o pojedynczych pierwiastkach $y_1,..., y_n\in\C$, więc $|\pi^{-1}(x_0)|=n$.

  Wokół każdego $(x_0, y_i)$ zbiór $Reg$ jest wykresem funkcji

  {\color{red}
    oj coś tutaj skipnęłam z $D$ hehe
    jest w zdjęciu
  }

  Niech $x_1\in D$. Wtedy rozwiązaniami równania $p(x_1, y)=0$ są $y_1(x_1), y_2(x_1),..., y_n(x_1)$ są parami różne i jest ich dokładnie $n$, czyli stopień $p(x_1, y)$ - więc to są wszystkie rozwiązania. Stąd $\pi^{-1}(x_1)\subseteq\bigcup\widetilde{D_i}$.
\end{proof}

Niech $s\in S$. Rozważmy $B(s, \epsilon)$ - koło wokół $s$ - rozłączne z $S-\{s\}$. Niech $B^*(s)=B(s, \epsilon)-\{s\}$ (małe koło wokół $s$ nakłóte w $s$). Wtedy $\pi:\pi^{-1}(B*(s))\to B*(s)$ jest nakryciem (być może niespójnym). 

Niech $B_j*(s)$ będzie składową spójną tego przeciwobrazu.
diagram boże jak mi sie dzisiaj nie chce pisac

\begin{lemma}{nakryciowy}{}
  Istnieje homeomorfizm $f_{s,j}$ zamykający diagram

  Ten $f_{s,j}$ jest biholomorfizmem, bo $\pi$, $z\mapsto z^k$ są lokalnymi boholomorfizmami
\end{lemma}

o jesuuu obrazeeeeeeeeeek

Dla każdego $B_j*(s)$ dokładamy punkcik $\hat{s_j}$. $f_{s,j}$ rozszerzamy deklarując $f_{s,j}(\hat{s_j})$ i dostajemy mapę $B_j*(s)\cup\{\hat{s_j})=:B_j(s)\to B(0, \epsilon^{1/k})$.

W ten sposób zbudowaliśmy tzw. powierzchnię Riemanna $\Sigma(p)\supseteq Reg$ funkcji algebracznej określonej przez $p$.

\begin{theorem}{}{}
  \begin{enumerate}
    \item $\Sigma(p)$ jest zwarta
    \item odwzorowanie $Reeg\ni (x,y)\mapsto y$ określa meromorficzną funkcję na $\Sigma(p)$
    \item $Reg$ jest spójny
    \item Niech $Z_{proj}(p)$ - projektywne uzwarcenie, czyli zbiór zer ujednorodnionego $p$

      Wtedy włożenie $Reg\hookrightarrow Z_{proj}(p)$ rozszerza się do holomorficznej surjekcji $\Sigma(p)\to Z_{proj}(p)$.
  \end{enumerate}
\end{theorem}

Tutaj przykład i obrazek

\begin{proof}
  \begin{enumerate}
    \item Pokryjemy $\Sigma(p)$ skończoną liczbą domkniętych dysków. Wokół każdego $s\in S$ wybieramy mały domknięty dysk $D_s$, którego przeciwobraz w $\Sigma(p)$ to skończona suma domkniętych dysków $D_j(s)$ - domknięte otoczenie $\hat{s_j}$.

      Niech $V=\overline{\C}-\bigcup_{s\in S}D(s)$, wtedy $\overline{V}$ jest zwartym podzbiorem $\overline{\C}-S$. Dla każdego $x\in\overline{V}$ wybieramy dysk $D(x)\subseteq\overline{\C}-S$ prawidłowo nakryty
      
      kolejne zdjęcie
  \end{enumerate}
\end{proof}



  
\end{document}
