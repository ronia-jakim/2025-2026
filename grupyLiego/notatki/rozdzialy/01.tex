\section{Grupa Liego}

\begin{definition}{grupa Liego}{}
  ...
\end{definition}

\begin{example}[m]
  \item grupy macierzowe (i ich domknięte podgrupy)
  \item grupa Heisenberga, czyli $\R^3$ z $(x_1, x_2, x_3)(y_1, y_2, y_3)=(x_1+y_1, x_2+y_2, x_3+y_3+\frac{1}{2}(x_1y_2-x_2y_1))$ ?? górnotrójkątne macierze o wyznaczniku $1$
  \item $\R^n$ z dodawaniem
  \item $\R^2$ z mnożeniem $(x_1, x_2)(y_1, y_2)=(x_1+y_1, e^{y_1}x_2+y_2)$ - to jest grupa rozwiązalna
  \item izometrie $\R^2$
\end{example}

Algebra Liego $\R$ z dodawaniem to $\R$ z dodawaniem i trywialna operacja nawiasu Liego (zawsze daje zero). Okrąg, czyli liczby rzeczywiste modulo liczby całkowite również ma trywialny nawias Liego.

Jedna największa spójna tego typu (uniwersalna) i inne grupy można uzyskać wydzielając ją przez grupy dyskretne.

grupa prosta - nietrywialne centrum

większa klasa grup półprostych
algebra półprosta jest sumą prostą algebr prostych

grupa nilpotentnna - Heisenberg jest najprostszy

Różniczkowanie - operacja spełniająca wzór Leibniza:
$$X(f\cdot g) = (Xf)\cdot g+f\cdot (Xg)$$

Lewe przesunięcie:
$(L_gf)(x)=f(g^{-1}x)$



